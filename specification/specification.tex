\documentclass[a4paper]{article}
\usepackage{cite}
\usepackage{listings}
\lstset{basicstyle=\ttfamily}
\usepackage{booktabs}

\title{Distinguish Between Memory Addresses and Data\\Thesis Specification}

\author{Xiaoyue Chen \and Stefanos Kaxiras}

\begin{document}
\maketitle

\section{Background}
% Here you describe in what context your thesis is to be done. What
% prerequisites are valid, what is the goal of the project from the
% supervisors point of view, what is available and has been done
% before, under what circumstances should the work be done.
Most modern computers do not distinguish between memory addresses and
data. Any content in the memory could be used directly or indirectly
as an address to access the memory. Even if the content is just data
(e.g., a password hash), it could still be used to reference the
memory (e.g., as an array index). While this property makes the
architecture more flexible, it also raises security concerns. By
exploiting this property, attackers could break memory isolation
between processes and leak confidential information.

\subsection{Spectre and Meltdown}
Spectre \cite{kocher2019spectre} and Meltdown \cite{lipp2018meltdown}
relies on using secret data to reference the memory. In both attacks,
attackers first access the secret data which will cause invalid
address accessing exception. Due to modern processors' out-of-order
execution and speculative execution optimisations, transient
instructions whose architectural effects will not be committed are
also executed.

In the transient instructions, attackers access the memory by using
the secret data as memory addresses. Although the transient
instructions will have no architectural effects, the memory accesses
would leave microarchitectural side effects, i.e., the referenced
memory is stored in the cache. Finally, attackers could use a cache
side channel to observe the side effects and read the secret data by
exploiting timing differences.

\subsection{Effects of memory addresses/data distinction}
The attacks introduced above could be mitigated if memory addresses
and data are distinguished. Using data as memory addresses could be
forbidden. As a result, the transient instructions in the attacks
would leave no microarchitectural side effects. This will improve the
security of computer systems.

\subsection{Goal}
Explore the techniques to distinguish between memory address and data
at different levels, e.g., microarchitectural level, assembly level,
compiler level.

\section{Description of the task}

\section{Methods}

\section{Relevant courses}
\begin{itemize}
\item Advanced Computer Architecture
\item Computer Architecture I
\item Compiler Design I
\item Operating Systems I
\item Secure Computer Systems I
\end{itemize}

\section{Delimitations}

\section{Time plan}
Xiaoyue Chen will be working full-time on the project for the entire
spring term in 2022. The detailed time plan follows table
\ref{tab:timeplan}.
\begin{table}[h]
  \centering
  \begin{tabular}{ll}
    \toprule
    Date &  Plan \\
    \midrule
    12 Jan & Project starts \\
    12 Jun & Project ends \\
    \bottomrule
  \end{tabular}
  \caption{The time plan for the project}
  \label{tab:timeplan}
\end{table}


\bibliographystyle{IEEEtran} \bibliography{references}

\end{document}

%%% Local Variables:
%%% mode: latex
%%% TeX-master: t
%%% End:
